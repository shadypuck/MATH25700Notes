\documentclass[../psets.tex]{subfiles}

\pagestyle{main}
\renewcommand{\leftmark}{Problem Set \thesection}
\setcounter{section}{7}

\begin{document}




\section{\texorpdfstring{$\bm{p}$}{}-Sylows and Simple Groups}
\begin{enumerate}
    \item \marginnote{12/2:}Show that the 2-Sylow subgroups of $S_4$ and $S_5$ are isomorphic to $D_8$, and the 2-Sylow subgroups of $A_4$ and $A_5$ are isomorphic to the Klein 4-group.
    \begin{proof}
        Conjugate subgroups are isomorphic, so we need only find one representative 2-Sylow of $S_4,S_5,A_4,A_5$ and work with each of them. Let's begin.\par
        For $S_4$, we have $4!=24=2^3\cdot 3$, and for $S_5$, we have $5!=120=2^3\cdot 15$. Thus, in both cases, we're looking for a subgroup of order 8, and the following will suffice.
        \begin{equation*}
            H = \{e,(13),(24),(12)(34),(13)(24),(14)(23),(1234),(4321)\}
        \end{equation*}
        Noting that $H=\gen{(1234),(13)}$ and $D_8=\gen{r,s}$, where $|(1234)|=|r|=4$ and $|(13)|=|s|=2$, we can define our isomorphism $\varphi:H\to D_8$ by
        \begin{align*}
            (1234) &\mapsto r&
            (13) &\mapsto s
        \end{align*}
        Everything else follows homomorphically.\par
        Similarly, for $A_4$, we have $12=2^2\cdot 3$ and for $A_5$, we have $60=2^2\cdot 15$. Thus, we're looking for a subgroup of order 4 this time, and the following will suffice.
        \begin{equation*}
            H = \{e,(12)(34),(13)(24),(14)(23)\}
        \end{equation*}
        Here, we define our isomorphism by
        \begin{align*}
            e &\mapsto e&
            (12)(34) &\mapsto (1,0)&
            (13)(24) &\mapsto (0,1)&
            (14)(23) &\mapsto (1,1)
        \end{align*}
    \end{proof}
    \item Let $H$ be the subset of $\text{GL}_3(\F_p)$ of matrices of the form
    \begin{equation*}
        \begin{pmatrix}
            1 & x & y\\
            0 & 1 & z\\
            0 & 0 & 1\\
        \end{pmatrix}
    \end{equation*}
    \begin{enumerate}
        \item Prove that $H$ is a $p$-Sylow subgroup of $\text{GL}_3(\F_p)$.
        \begin{proof}
            We know from \textcite[35]{bib:DummitFoote} that
            \begin{align*}
                |\text{GL}_3(\F_p)| &= (p^3-1)(p^3-p)(p^3-p^2)\\
                &= p^9-p^8-p^7+p^5+p^4-p^3\\
                &= p^3\cdot(p^6-p^5-p^4+p^2+p-1)
            \end{align*}
            Additionally, each variable $x,y,z$ in the prototypical element of $H$ can take on all $p$ possible values without affecting the status of that matrix as an element of $\text{GL}_3(\F_p)$. This is because that (upper triangular) matrix's determinant will always be the product of its unchanging diagonal entries. Therefore, $|H|=p^3$. It follows by the definition of $p$-Sylows that $H$ is a $p$-Sylow of $\text{GL}_3(\F_p)$, as desired.
        \end{proof}
        \item Prove that $H$ is not normal.
        \begin{proof}
            To prove that $H$ is not normal, it will suffice to find $h\in H$ and $g\in G$ such that $ghg^{-1}\notin H$. Indeed, if we take
            \begin{align*}
                h &=
                \begin{pmatrix}
                    1 & 0 & 0\\
                    0 & 1 & 2\\
                    0 & 0 & 1\\
                \end{pmatrix}&
                g &=
                \begin{pmatrix}
                    1 & 0 & 0\\
                    0 & 0 & 1\\
                    0 & 1 & 0\\
                \end{pmatrix}
            \end{align*}
            then
            \begin{equation*}
                \underbrace{
                    \begin{pmatrix}
                        1 & 0 & 0\\
                        0 & 0 & 1\\
                        0 & 1 & 0\\
                    \end{pmatrix}
                }_g\underbrace{
                    \begin{pmatrix}
                        1 & 0 & 0\\
                        0 & 1 & 2\\
                        0 & 0 & 1\\
                    \end{pmatrix}
                }_h\underbrace{
                    \begin{pmatrix}
                        1 & 0 & 0\\
                        0 & 0 & 1\\
                        0 & 1 & 0\\
                    \end{pmatrix}
                }_{g^{-1}}
                =
                \begin{pmatrix}
                    1 & 0 & 0\\
                    0 & 1 & 0\\
                    0 & 2 & 1\\
                \end{pmatrix}
                \notin H
            \end{equation*}
            as desired.
        \end{proof}
        \item Determine the number $n_p$ of $p$-Sylow subgroups of $\text{GL}_3(\F_p)$.
        \begin{proof}
            Prove 2d and then by Sylow III, take
            \begin{equation*}
                n_p = [G:N_G(H)]
            \end{equation*}
            From
            \begin{equation*}
                N_G(H) = \left\{
                    \begin{pmatrix}
                        * & * & *\\
                        0 & * & *\\
                        0 & 0 & *\\
                    \end{pmatrix}
                    \mid
                    \in\text{GL}_3(\F_p)
                \right\}
            \end{equation*}
            That $|N_G(H)|=(p-1)^3p^3$, $|G|=|\text{GL}_3(\F_p)|$. Recall that elements of $\text{GL}_3(\F_p)$ lives in three columns. Treat the columns one by one. The number of choices for the first are $p^3-1$. The number of choices for the second are $p$ multiples of the first column in $\F_p^3$, $p^3-p$ chosen for the second column. There will be $p^3-p^2$ choices for the third column. Thus, there are $(p^3-1)(p^3-p)(p^3-p^2)$ ways to choose the columns; this is the order of $|\text{GL}_3(\F_p)|$.
            Implies that the order
            \begin{align*}
                n_p &= [G:N_G(H)]\\
                &= \frac{|G|}{|N_G(H)|}\\
                &= \frac{(p-1)^3p^3(p^2+p+1)(p+1)}{(p-1)^3p^3}\\
                &= (p^2+p+1)(p+1)
            \end{align*}
            This is a very important computation and Abhijit wants to make sure we really understand it! Write something about it in my OH notes.\par
            If we ever learn Rep theory, we'll learn a different proof of this idea. Denote by $U$ the set of upper triangular matrices. Our proposition 1 is that $N_H(G)=U$. Proposition 2 is $N_U(G)=U$. Why does 2 imply 1? It turns out that $H\triangleleft U$. This is rather subtle. We want to show that $N_G(H)=U$, where $H$ is the \textbf{Heisenberg group of matrices}. Check $U\subset N_G(H)$. Approach 1: "Do it" with matrix multiplication and cogue that the diagonal of $ghg^{-1}$ is all ones if $g\in U$. Approach 2: Conjugation in matrix groups is a change of basis. Conjugating by $BmB^{-1}$ is a change of basis from $\{e_1,\dots,e_n\}\mapsto\{Be_1,\dots,Be_n\}$. This does not change how the operator/matrix acts on subspaces. Recall that much of linear algebra can be done in a basis-free sense.
        \end{proof}
        \item Determine the normalizer of $H$.
    \end{enumerate}
    \item Suppose that $P$ is a normal $p$-Sylow subgroup of $G$. Suppose that $H$ is a subgroup of $G$. Prove that $P\cap H$ is the unique $p$-Sylow subgroup of $H$. (Exercise 4.5.33 of \textcite{bib:DummitFoote}.)
    \begin{proof}
        % $P$ is normal, so $gPg^{-1}=P$ for all $g\in G$. Sylow II: $P$ is the unique $p$-Sylow subgroup of $G$. Sylow III: $n_p=1$. $P\cap H\subset H$.

        % $|H|=p^m\cdot k$, $|G|=p^{m+n}\cdot kk'$.

        % Sylow I: $H$ contains a $p$-Sylow.

        % $P\cap H$ is a maximal $p$-subgroup of $H$ (since $P$ is the unique maximal $p$ subgroup of $G$). $P\cap H\leq P$ and subgroups of $p$-groups are $p$-groups.


        % $g(P\cap H)g^{-1}=P\cap H$?? Let $x\in P\cap H$, and let $h\in H$. Then $hxh^{-1}\in P$ since $P$ is normal. Additionally, since $h,x\in H$, naturally the product $hxh^{-1}\in H$. Therefore, $P\cap H\triangleleft H$, so if $P\cap H$ is a $p$-Sylow, it is unique.

        % $P\cap H$ is a subgroup of $P$, so $|P\cap H|\mid|P|$ by Lagrange's theorem.

        % $|PH|/|P|=|H|/|P\cap H|$. $|P\cap H|=|H||P|/|PH|$.

        % $P$ is the unique maximal $p$-subgroup of $G$. This means that no $p$-group contains $P$ strictly. Thus, $P\cap H$ must be a maximal $p$-subgroup of $G$. Suppose for the sake of contradiction that $P\cap H\subsetneq P'$, where $P'$ is a $p$-Sylow of $H$. $P\cap H\leq P$ implies that $P\cap H$ is a $p$-group.


        To prove that $P\cap H$ is the unique $p$-Sylow of $H$, we must show that $P\cap H$ is a $p$-Sylow of $H$ and that $P\cap H\triangleleft H$. Let's begin.\par
        Since $P$ is a normal $p$-Sylow, Sylow II implies that $P$ is the only $p$-Sylow in $G$. Thus, all $p$-groups in $G$ are subgroups of $P$. In particular, since $P\cap H\leq P$, $P\cap H$ is a $p$-group and, moreover, it must be the maximal $p$-group (or $p$-Sylow) in $H$ since any larger $p$-group would by definition necessarily have elements lying outside of $H$.\par
        To prove that $P\cap H\triangleleft H$, it will suffice to show that $P\cap H\subset H$ and if $h\in H$ and $x\in P\cap H$, $hxh^{-1}\in P\cap H$. The first claim clearly follows from the set theoretic definition of the intersection. For the second claim, we know that $x\in P$ since $x\in P\cap H$. Thus, since $P$ is normal in $G$ and $h\in H\subset G$, $hxh^{-1}\in P$. Additionally, since $x,h\in H$ and $H$ is a subgroup, we know that the product $hxh^{-1}\in H$. But if $hxh^{-1}\in P,H$, then $hxh^{-1}\in P\cap H$, as desired.
    \end{proof}
    \item Prove that if $n<p^2$, the $p$-Sylow subgroup of $S_n$ is abelian. Prove that if $n\geq p^2$, the $p$-Sylow subgroup of $S_n$ is \emph{not} abelian.
    \begin{proof}
        % What changes at that divide? We go from having $p^{0,1}$ in the prime factorization to having $p^{2+}$.

        Groups of order $p^2$ and groups of order $p$ are abelian, always?? Counterexample: $p=3$, $S_9$ has abelian $p$-Sylow
        \begin{equation*}
            \gen{(1,2,3,4,5,6,7,8,9)}
        \end{equation*}
    \end{proof}
    \item Let $N$ be a normal subgroup of $G$, and suppose that the largest power of $p$ dividing $|N|$ is equal to the largest power of $p$ dividing $|G|$. Prove that the $p$-Sylow subgroups of $G$ are precisely the $p$-Sylow subgroups of $N$.
    \begin{proof}
        % $|N|=p^m\cdot k$, $|G|=p^m\cdot kk'$.
        % $gNg^{-1}=N$ for all $g\in G$.

        % $G$ has a $p$-Sylow. The $p$-Sylows of $G$ are all conjugate to each other.
        % $N$ has a $p$-Sylow. It will be a $p$-Sylow of $G$ as well. All of the $p$-Sylows of $N$ are conjugate to each other. Are there $p$-Sylows of $G$ not contained in $N$?

        Every $p$-Sylow of $N$ is a $p$-Sylow of $G$.
        Suppose for the sake of contradiction that there exists a $p$-Sylow $Q\subset G$ such that $Q\not\subset N$.
        Let $P$ be a $p$-Sylow of $N$ (guaranteed to exist by Sylow I).
        Sylow II: There exists $g\in G$ such that $gPg^{-1}=Q$. In particular, let $q\in Q$ be such that $q\notin N$. Then $q=gpg^{-1}$ for some $p\in P\subset N$. But this implies that not all $p\in N$ satisfy $gpg^{-1}\in N$, a contradiction.
    \end{proof}
    \item Prove that there do not exist any simple groups of order $p^2q$ for distinct primes $p,q$. (\emph{Hint}: Consider the congruence restrictions from Sylow III.)
    \begin{proof}
        Let $G$ be a group of order $|G|=p^2q$ for $p,q$ distinct primes. Suppose for the sake of contradiction that $G$ is simple. We divide into two cases ($p>q$ and $p<q$).\par
        First, let $p>q$. Sylow III: $n_p\equiv 1\bmod p$ and $n_p\mid q$. Thus, $n_p\in\{1,q\}$. If $n_p=1$, we are done. If $n_p=q$, then $n_p\not\equiv 1\bmod p$, a contradiction.\par
        Second, let $p<q$. Sylow III: $n_q\equiv 1\bmod q$ and $n_q\mid p^2$. Thus, $n_q\in\{1,p,p^2\}$. If $n_q=1$, we are done. If $n_q=p$, then $n_q\not\equiv 1\bmod q$. If $n_q=p^2$, then the total number of elements of order $q$ is $n_q(q-1)=p^2(q-1)=p^2q-p^2$. Thus, only $p^2$ elements of $G$ do not have order $q$. But since by Sylow I there must exist a $p$-Sylow of order $p^2$ in $G$, these remaining elements will be used up by that $p$-Sylow. Since there are no more element of $G$, there is only one $p$-Sylow in $G$, which is necessarily normal, a contradiction.
    \end{proof}
    \item Prove that there do not exist any simple groups of the following orders. (Warning: Not in order of difficulty.)
    \begin{enumerate}
        \item (*) 336.
        \item 1176.
        \begin{proof}
            $1176=2^3\cdot 3\cdot 7^2$. We have $n_7\equiv 1\bmod 7$ and $n_7|24$. Thus, $n_7=1,8$. If $n_7=1$, we are done. Now suppose $n_7=8$.
        \end{proof}
        \item 2907.
        \begin{proof}
            $2907=3^2\cdot 17\cdot 19$.
        \end{proof}
        \item 6545.
        \begin{proof}
            $6545=5\cdot 7\cdot 11\cdot 17$.
        \end{proof}
    \end{enumerate}
\end{enumerate}




\end{document}