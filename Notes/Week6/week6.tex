\documentclass[../notes.tex]{subfiles}

\pagestyle{main}
\renewcommand{\chaptermark}[1]{\markboth{\chaptername\ \thechapter\ (#1)}{}}
\setcounter{chapter}{5}

\begin{document}




\chapter{Fundamentals of Group Actions}
\section{Examples of Group Actions}
\begin{itemize}
    \item \marginnote{10/31:}Today: A number of interesting group actions.
    \item \textbf{Left action} (of $G$ on $X$): A group action of the form $g\cdot x$ (as opposed to $x\cdot g$).
    \item Let $G$ be a group, and let $X=G$. Take $g\cdot x=gx$.
    \begin{itemize}
        \item Axiom confirmation.
        \begin{enumerate}
            \item $e\cdot x=ex=x$.
            \item $g\cdot(h\cdot x)=ghx=gh\cdot x$.
        \end{enumerate}
        \item Let $e\in X$. Then $\Orb(e)=X$. In particular, this means that the action is transitive.
        \item $\Stab(x)=\{g\in G\mid gx=x\}=\{e\}$ for $x\in X$ arbitrary, in general.
        \item $\ker=\{e\}$. This also follows from the above. Thus, the action is faithful.
    \end{itemize}
    \item Corollary: Let $G$ be a finite group. Then $G$ is isomorphic to a subgroup of $S_n$ for some $n$. We may take $n=|G|$.
    \begin{itemize}
        \item Construction: We invoke the proposition from last lecture. In particular, we know that the action $G\acts G$ implies the existence of a homomorphism $\phi:G\to S_G$ defined by $g\mapsto\psi_g$.
        \item The map in the above construction has trivial kernel. By the FIT, $G/\ker\cong\im\phi$. Combining these results, we obtain $G\cong G/\ker\cong\im\phi\leq S_n$.
        \item Applying this construction to $S_3$, we deduce that $S_3\leq S_6$.
    \end{itemize}
    \item $\text{SO}(2)\cong\R/\Z\cong\Q/\Z\oplus\Q^\infty$.
    \begin{itemize}
        \item In infinite cases, you usually want to consider some other topological things that disappear in the finite case.
    \end{itemize}
    \item Let $G$ be a group and take $X=G$ again. We can also consider $g\cdot x=gxg^{-1}$.
    \begin{itemize}
        \item Axioms.
        \begin{enumerate}
            \item $e\cdot x=exe^{-1}=x$.
            \item $g\cdot(h\cdot x)=ghxh^{-1}g^{-1}=(gh)x(gh)^{-1}=gh\cdot x$.
        \end{enumerate}
        \item $\Orb(e)=\{e\}$; not transitive if $|G|>1$.
        \item Let $x\in X$. Then $\Orb(x)$ is the conjugacy class of $x$.
        \item $\Stab(x)=C_G(x)$.
        \item $\ker=Z(G)$. Thus, the group action is faithful iff the center is trivial. Abelian implies not faithful.
    \end{itemize}
    \item A nice thing about these constructions is that they cast other constructions we've encountered in the more general language of group actions.
    \item \textbf{Right actions} are even nastier than left cosets and right cosets, so Calegari will not mention them again.
    \begin{itemize}
        \item $g\cdot x=x\cdot g^{-1}$ and $g\cdot(h\cdot x)=(x\cdot h^{-1})\cdot g^{-1}$.
    \end{itemize}
    \item Let $G=G$, $X$ be the subgroups of $G$. $g\cdot H=gHg^{-1}$.
    \begin{itemize}
        \item Note that $H\leq G$ does indeed imply that $gHg^{-1}\leq G$. In particular, \dots
        \begin{itemize}
            \item $H$ is nonempty (contains at least $e$), so $gHg^{-1}\supset\{geg^{-1}\}$ is nonempty;
            \item $gh_1g^{-1},gh_2g^{-1}\in gHg^{-1}$ imply that $gh_1g^{-1}gh_2g^{-1}=g(h_1h_2)g^{-1}\in gHg^{-1}$;
            \item $ghg^{-1}\in gHg^{-1}$ has inverse $gh^{-1}g^{-1}\in gHg^{-1}$.
        \end{itemize}
        \item Axioms (entirely analogous to the last example).
        \item $\Orb(H)$ is the "conjugates" of $H$.
        \item $\Stab(H)=N_G(H)$.
        \item $\ker=?$. We know that $Z(G)\subset\ker$. The conclusion is that there is not a nice definition for the kernel other than the intersections of the stabilizers/normalizers.
        \begin{itemize}
            \item ...
            \item If any $H\triangleleft G$ is normal, and $x\in G$ had order 2, then $\gen{x}\triangleleft G$, meaning that $gxg^{-1}\in\gen{x}$, i.e., $x\in Z(G)$, so this rules out $D_8$??
        \end{itemize}
    \end{itemize}
    \item Fix $G$ and $H\leq G$. Let $X=G/H$ (not assuming $H\triangleleft G$, so we know that $G/H$ is the set of left cosets but it is not a group in general). Define $g\cdot xH=gxH$.
    \begin{itemize}
        \item We have $g\cdot xhH=gxhH$.
        \item Orbit: $\Orb(eH)=X$.
        \item Stabilizer: $\Stab(eH)=H$.
        \begin{itemize}
            \item $\Stab(gH)=gHg^{-1}$.
            \item This is because $(ghg^{-1})gH=ghH=gH$.
            \item Go to the more general case $G\acts X$, $\Stab(x)=H$. Then $gHg^{-1}\subset\Stab(g\cdot x)$??
        \end{itemize}
        \item Transitive: Yes (see orbits).
        \item Faithful: If $H$ is normal, no. If $H$ contains a normal subgroup, no. Maybe yes.
        \item Kernel: If $H$ is normal, then $\ker=H$. In general, $\ker=\bigcap_{g\in G}gHg^{-1}$ (the largest normal subgroup of $H$).
    \end{itemize}
    \item Takeaway: General constructions allow us to see things we've already done.
    \item Next time: The most useful theorem of the course, that provides lots of information on relations between objects.
\end{itemize}



\section{Orbit-Stabilizer Theorem}
\begin{itemize}
    \item \marginnote{11/2:}We will have a take-home open-book final. Should take you a couple hours or a little more to do, but we'll have more time than that. Don't Google answers or collaborate. We'll have more practice problems (and 50\% of the exam will be on that sheet); if we do every problem on the sheet, we'll certainly get an A.
    \item We will cover all theoretical material by Thanksgiving and then spend the rest of the time exploring applications.
    \item Today: The most fundamental theorem of the class.
    \item Let $G$ be a group acting on a set $X$.
    \item Theorem (Orbit-Stabilizer Theorem): Let $x\in X$ be arbitrary. Then
    \begin{equation*}
        |G| = |\Orb(x)|\cdot|\Stab(x)|
    \end{equation*}
    \begin{proof}
        We will break up $G$ and count it in two different ways. Let $x\in X$ be arbitrary and consider $\Orb(x)$. By definition, $\Orb(x)$ is the set of all $y$ such that $g\cdot x=y$ for some $g\in G$. Equivalently, every $g\in G$ maps $x$ to some $y\in\Orb(x)$. Thus, we can partition $G$ into sets of $g$ that map $x$ to a particular $y$, knowing that every $g$ must send it to some $y$. Symbolically,
        \begin{equation*}
            G = \bigsqcup_{y\in\Orb(x)}\{g\mid g\cdot x=y\}
        \end{equation*}
        Each of the sets over which we sum above is equal to $g\cdot\Stab(x)$ (the left coset of the stabilizer by $g$).\par
        Thus, for each $y\in\Orb(x)$, we contribute $|g\cdot\Stab(x)|$ to $|G|$. Symbolically,
        \begin{equation*}
            |G| = \sum_{\Orb(x)}|g\cdot\Stab(x)|
            = \sum_{\Orb(x)}|\Stab(x)|
            = |\Orb(x)|\cdot|\Stab(x)|
        \end{equation*}
        as desired.
    \end{proof}
    \item Examples:
    \begin{itemize}
        \item Let $H\leq G$, $X=G/H$. Then $G$ acts on $X$ by left multiplication. Taking $x=H$ in particular, we have that
        \begin{equation*}
            |G| = |G/H|\cdot|H|
        \end{equation*}
        and we recover Lagrange's theorem as a special case of the O-S theorem.
        \item $G=S_n$, $X=[n]$.
        \begin{itemize}
            \item Then $S_n=\{\sigma(1)=1\}\cup\{\sigma(1)=2\}\cup\cdots\cup\{\sigma(1)=n\}$. This is analogous to the proof strategy decomposition.
        \end{itemize}
        \item $G$ acts on $G$ by conjugation.
        \begin{itemize}
            \item Take $g\in G$. Then $\Orb(g)=\{g\}$, i.e., the conjugacy class of $g$, and $\Stab(g)=C_G(g)$. Therefore, we have the below corollary.
        \end{itemize}
        \item $G=S_n$.
        \begin{itemize}
            \item Let $g=(1,\dots,k)$ for $2\leq k\leq n$. Recall that $|\{g\}|=n!/(n-k)!k$. Thus, $|C_{S_n}(g)|=(n-k)!\cdot k$.
            \item Alternatively, we can derive the order of this centralizer directly: $C_{S_n}(g)=\gen{g}\times S_{n-k}$, i.e., all powers of the $k$-cycle and everything that's disjoint. $\times$ denotes the direct product.
        \end{itemize}
        \item $G=S_4$, $g=(12)(34)$.
        \begin{itemize}
            \item $|\{g\}|=3$, so $|C_G(g)|=8$.
            \item Here $C_G(g)=D_8$. Visualize a square with vertices clockwise (1,4,2,3).
        \end{itemize}
        \item $G=S_6$, $g=(16)(25)(34)$.
        \begin{itemize}
            \item We have that $|\{g\}|=6!/2^3\cdot 3!=15$, so $|C_{S_6}(g)|=48$. The centralizer is the set of all elements satisfying $\sigma(i)+\sigma(7-i)=7$.
            \item Moreover, there is an injective homomorphism from $\widetilde{\text{Cu}}\hookrightarrow S_6$ whose image is exactly the centralizer of $(16)(25)(34)$. Moreover, it follows that $C_{S_6}(g)\cong S_4\times S_2$.
            \item Let $h=(16)$. Then $|\{h\}|=|\{g\}|=15$. Does there exist an automorphism of $S_6$ to $S_6$ which sends $h\to g$? No: $S_2\times S_4\cong C_{S_6}(h)$ and $C_{S_6}(g)\cong S_2\times S_4$.
        \end{itemize}
    \end{itemize}
    \item Corollary: We have that
    \begin{equation*}
        |G| = |\{g\}|\cdot|C_G(g)|
    \end{equation*}
    \item $\bm{\widetilde{\textbf{Cu}}}$: The set of all orthogonal symmetries of the cube (i.e., including reflections).
    \begin{itemize}
        \item There is an isomorphism between $\text{Cu}\times\Z/2\Z$ and $\widetilde{\text{Cu}}$ defined by $(g,1)\mapsto g$ and $(g,-1)\mapsto -g$. The reverse function is $g\mapsto(g\cdot\deg g,\deg g)$.
        \item $\widetilde{\text{Cu}}$ acts on 6 faces.
    \end{itemize}
    \item The pace will be this fast through Thanksgiving.
\end{itemize}



\section{Blog Post: The Orbit-Stabilizer Theorem, Cayley's Theorem}
\emph{From \textcite{bib:Calegari}.}
\begin{itemize}
    \item \marginnote{11/13:}Lemma: Let $G\acts X$ and let $x\in X$. Let $y\in\Orb(x)$, i.e., let there exist $\sigma\in G$ such that $y=\sigma\cdot x$. Then
    \begin{enumerate}
        \item $\Stab(y)=\sigma\cdot\Stab(x)\cdot\sigma^{-1}$.
        \begin{proof}
            Let $H:=\Stab(x)$. We use a bidirectional inclusion argument.\par
            Suppose first that $\sigma h\sigma^{-1}\in\sigma H\sigma^{-1}$. Then
            \begin{equation*}
                \sigma h\sigma^{-1}\cdot y = \sigma h\cdot(\sigma^{-1}\cdot y)
                = \sigma h\cdot x
                = \sigma\cdot(h\cdot x)
                = \sigma\cdot x
                = y
            \end{equation*}
            so $\sigma h\sigma^{-1}\in\Stab(y)$, as desired.\par
            Now suppose that $g\in\Stab(y)$. An analogous argument to the above shows that $\sigma^{-1}g\sigma\in\Stab(x)$, so $g=\sigma h\sigma^{-1}\in\sigma H\sigma^{-1}$, as desired.
        \end{proof}
        \item The set of elements $g\in G$ such that $g\cdot x=y$ is exactly the coset $\sigma\cdot\Stab(x)$.
        \begin{proof}
            As before, let $H:=\Stab(x)$ and proceed via a bidirectional inclusion argument.\par
            Suppose first that $\sigma h\in\sigma H$. Then
            \begin{equation*}
                \sigma h\cdot x = \sigma\cdot(h\cdot x)
                = \sigma\cdot x
                = y
            \end{equation*}
            so $\sigma h$ is in the first set, as desired.\par
            Now suppose that $g\cdot x=y$. Since $\sigma\cdot x=y$ as well by hypothesis, it follows by transitivity that
            \begin{align*}
                g\cdot x &= \sigma\cdot x\\
                \sigma^{-1}\cdot(g\cdot x) &= \sigma^{-1}\cdot(\sigma\cdot x)\\
                \sigma^{-1}g\cdot x &= x
            \end{align*}
            This implies that $\sigma^{-1}g\in H$, i.e., that $g\in\sigma H$, as desired.
        \end{proof}
    \end{enumerate}
    \item This lemma further justifies the following step we took when proving the Orbit-Stabilizer Theorem in class: Equating each $\{g\mid g\cdot x=y\}=\sigma\Stab(x)$.
    \item Further comments on $G\acts G/H$ ($H$ a subgroup).
    \begin{itemize}
        \item Why the action is well-defined.
        \begin{itemize}
            \item $g\cdot xhH=gxhH=gxH=g\cdot xH$.
            \item What saves the day here is that we're combining an unambiguous term ($g$) with our ambiguous term ($xH$) instead of trying to combine two ambiguous terms (e.g., $xH$ and $yH$).
        \end{itemize}
        \item An example where the action is faithful.
        \begin{itemize}
            \item Let $G=S_n$ and $H=\{\sigma\mid\sigma(1)=1\}\cong S_{n-1}$.
            \item Note that if $\sigma\in H$, then $(1,k)\sigma(1,k)^{-1}$ sends $\sigma(k)=k$.
            \item Thus,
            \begin{equation*}
                \ker = \bigcap_{g\in G}gHg^{-1} 
                \subset \bigcap_{k=1,\dots,n}(1,k)H(1,k)^{-1}
                = \{e\}
            \end{equation*}
            so the action is faithful, here.
        \end{itemize}
        \item When $H=\{e\}$, $G\acts G/H$ is entirely analogous to left multiplication within the group: $g\cdot x=gx$.
    \end{itemize}
    \item Lemma: $G\acts G$ by left multiplication is faithful.
    \begin{proof}
        To prove this result, we will actually prove the stronger result that $\Stab(x)=\{e\}$ for all $x\in G$, from which it will follow that $\ker=\bigcap_{x\in G}\Stab(x)=\{e\}$. We have this stronger result by the cancellation lemma since
        \begin{align*}
            g\cdot x &= x\\
            gx &= ex\\
            g &= e
        \end{align*}
        for all $g\in\Stab(x)$.
    \end{proof}
    \item Corollary (Cayley's Theorem): If $|G|=n$, then $G\leq S_n$.
    \begin{proof}
        From the construction $G\acts G$ via left multiplication, we get a homomorphism $\phi:G\to S_G$ as per the Proposition in Lecture 5.2. Since this action is faithful (by the lemma), this homomorphism is an injection. This implies that $G\cong\im\phi\leq S_G\cong S_n$, as desired.
    \end{proof}
    \item Implication: Even without knowing anything about $G$, we can get useful information by considering its actions on a set.
    \item More on $G\acts G$ by conjugation.
    \begin{itemize}
        \item Since $|G|=|\{g\}|\cdot|C_G(g)|$, we can calculate the orders of centralizers. From the order, we can often get even more specific information.
        \item Consider $G=S_n$.
        \begin{itemize}
            \item If $g=(1,2,\dots,n)$, then $|\{g\}|=(n-1)!$ and $C_G(g)=n!/(n-1)!=n$. This combined with the fact that $g$ commutes with $g$ implies that
            \begin{equation*}
                C_{S_n}((1,2,\dots,n)) = \gen{(1,2,\dots,n)}
            \end{equation*}
            \item If $g=(1,2,\dots,k)$, then $|\{g\}|=n!/k(n-k)!$ so $|C_{S_n}(g)|=k\cdot(n-k)!$. Naturally, $g\in C_{S_n}(g)$, but so are all elements which fix $1,2,\dots,k$ and shuffle $k+1,k+2,\dots,n$. Thus,
            \begin{equation*}
                C_{S_n}((1,2,\dots,k)) = \Z/k\Z\times S_{n-k}
            \end{equation*}
            \item Let $g$ have cycle shape corresponding to the partition $a_1n_1+a_2n_2+\cdots$ where $n_1>n_2>\cdots$ denote cycle lengths and the $a_i$ denote the corresponding multiplicity. We can deduce that the centralizer has order $\prod n_i^{a_i}a_i!$.\par
            It follows from the fact that disjoint cycles commute that $g$ commutes with each component cycle, i.e., if $g=\cdots(a_1,\dots,a_k)\cdots$, then $g$ and $(a_1,\dots,a_k)$ commute. $g$ therefore also commutes with all powers of each component cycle. Going even further, $g$ commutes with all products of all powers of each component cycle, i.e., if $g=(a_1,\dots,a_k)(b_1,\dots,b_\ell)(c_1,c_2,\dots)\cdots$, then
            \begin{equation*}
                C_{S_n}(g) \supset \gen{(a_1,\dots,a_k),(b_1,\dots,b_\ell),(c_1,c_2,\dots),\dots}
            \end{equation*}
            The group on the right above is isomorphic to $\prod(\Z/n_i\Z)^{a_i}$ and thus has order $\prod n_i^{a_i}$.\par
            What are the other elements in the centralizer that account for the $\prod a_i!$ term?? Is it the products of the powers of the cycles??
        \end{itemize}
        \item How many elements $g\in G$ make $g\cdot x=y$ true?
        \begin{itemize}
            \item Equivalent to asking how many $g\in G$ make $gxg^{-1}=y$.
            \item Relating to before, this will be a coset of the centralizer (we need a particular solution, and then we can compose it with all homogeneous solutions).
        \end{itemize}
    \end{itemize}
    \item More on $G\acts X$ ($X$ is the set of subsets of $G$).
    \begin{itemize}
        \item Let $H$ be a subgroup. Since $\Orb(H)$ is the conjugates of $H$ and $\Stab(H)=N_G(H)$, we have by the Orbit-Stabilizer Theorem that the number of subgroups of $G$ conjugate to $H$ is equal to $|G|/|N_G(H)|=[G:N_G(H)]$.
    \end{itemize}
\end{itemize}



\section{Group Actions on the Quotient Group}
\begin{itemize}
    \item \marginnote{11/4:}Let $G\supset H$ and $X=G/H$. Consider a group action $G\acts X$ defined by $g\cdot xH=gxH$ that is transitive.
    \item Recall that $xH=yH$ iff $x=yh$ for some $h\in H$ iff $y^{-1}x\in H$.
    \item Example: Consider $G=S_4$ and $H=D_8=\gen{(1234),(13)}$.
    \item Let $A=H$, $B=(123)H$, $C=(123)^2H$ be the three elements of $X=G/H=S_4/D_8$.
    \item We define a homomorphism $\phi:S_4\to S_X=S_{\{A,B,C\}}$ by
    \begin{equation*}
        \phi(\sigma) =
        \begin{cases}
            A &\mapsto \sigma A\\
            B &\mapsto \sigma B\\
            C &\mapsto \sigma C
        \end{cases}
    \end{equation*}
    \begin{itemize}
        \item Example: $\phi(123)=(ABC)$.
        \item Example: $\phi(1234)$ is the element of $S_{\{A,B,C\}}$ that sends $A\mapsto(1234)H=H=A$, $B\mapsto(1234)(123)H=(1324)H=C$, and $C\mapsto(1234)(132)H=(14)H=B$. Thus, $\phi(1234)=(BC)$.
        \item Let $x=(14)$ and $y=(123)$. Then $y^{-1}x=(321)(14)=(1432)=(1234)^{-1}\in H$, so $xH=yH$.
    \end{itemize}
    \item Investigating $\ker\phi$.
    \begin{itemize}
        \item $\phi((13)(24))=(BC)^2=e$. Thus, $(13)(24)\in\ker$ and it follows that everything conjugate to it is as well.
        \item By the FIT, $S_4/\ker\phi\cong S_3$ so $|\ker\phi|=4$.
        \item Thus, $\ker\phi=\{e,(12)(34),(13)(24),(14)(23)\}$.
    \end{itemize}
    \item Investigating the stabilizers on $X$.
    \begin{itemize}
        \item $\Stab(A)=H$.
        \begin{itemize}
            \item Naturally, every $h\in H$ makes $hH=H$.
        \end{itemize}
        \item $\Stab(B)=\Stab((123)H)=(123)H(123)^{-1}$.
        \begin{itemize}
            \item This is because any $(123)h(123)^{-1}\in(123)H(123)^{-1}$ makes
            \begin{equation*}
                (123)h(123)^{-1}(123)H = (123)hH
                = (123)H
            \end{equation*}
        \end{itemize}
        \item It follows by similar logic that $\Stab(C)=(132)H(132)^{-1}$.
    \end{itemize}
    \item Is something about $H$ special in determining this action?
    \begin{itemize}
        \item Suppose you take $H'=(123)H(123)^{-1}$. Is $G\acts G/H'$ the same action? The cosets of $H'$ are $(123)H'$ and $(132)H'$. Let $A'=(132)H'$, $B'=H'$, and $C'=(123)H'$.
        \item It follows that $A'=(132)(123)H(123)^{-1}=A(123)^{-1}$, $B'=(123)H(123)^{-1}=B(123)^{-1}$ and $C'=(123)(123)H(123)^{-1}=C(123)^{-1}$.
        \item Conclusion: Take $H,gHg^{-1}$. Let $A$ be a left coset of $H$. Then $Ag^{-1}$ is a left coset of $gHg^{-1}$.
        \item First, a coset (like $A$) is the set of all elements that send $x$ to $y$.
        \item Suppose $g\cdot x=z$. Then the coset is $Ag^{-1}$??
    \end{itemize}
    \item Take $G$ and $H=\{e\}$, $G\acts G$ the left matrices??
    \item Another example: Let $G=S_3=\{e,(123),(123)^2,(12),(12)(123),(12)(123)^2\}$.
    \item Again, we can define a homomorphism $\phi:G\to S_G$. Call the above elements of $S_3$ A-F, respectively, as listed above.
    \begin{itemize}
        \item Example: $\phi(123)=(ABC)(DFE)$.
        \item Example: $\phi(12)=(AD)(BE)(CF)$.
    \end{itemize}
    \item Let $|g|=k$, e.g., $g^{k=1}$ is distinct.
    \begin{itemize}
        \item $x,gx$ and $g^{k-1}x$ all distinct.
        \item The cycle class of $\phi(g)$ is all $k$-cycles where $k=|g|\big||G|$.
        \item The remark here is that if $|g|=k$, not only are $e,\dots,g^{k-1}$ distinct, but $x,\dots,g^{k-1}x$ are distinct.
    \end{itemize}
    \item Exotic automorphism of $S_6$.
    \item Take $S_5$, and let $X$ be the set of subgroups of $S_5$ of order 5. We may also call this the subgroups generated by 5-cycles.
    \item Let $S_5$ act on $X$ by conjugation.
    \item The action is transitive.
    \item $|X|=24/4=6$.
    \begin{itemize}
        \item There are $\binom{5}{5}(5-1)!=24$ elements of order 5, i.e., 5-cycles in $S_5$.
        \item Each subgroup of $S_5$ of order 5 contains 4 distinct 5-cycles and $e$.
        \item These remarks imply the above result.
    \end{itemize}
    \item Therefore, we get a map $\phi:S_5\to S_X$.
    \item Take $P=\gen{(12345)}$.
    \begin{itemize}
        \item We have
        \begin{equation*}
            \Stab(P) = \{g\in G\mid g\cdot P=P\}
            = \{g\in G\mid gPg^{-1}=P\}
            = N_{S_5}(P)
        \end{equation*}
        \item Since the action is transitive, $\Orb(P)=X$. Thus, by the Orbit-Stabilizer theorem,
        \begin{equation*}
            |N_{S_5}(P)| = \frac{|G|}{|X|}
            = \frac{120}{6}
            = 20
        \end{equation*}
    \end{itemize}
    \item $\ker\phi=\{\{e\},A_5,S_5\}$.
    \item By the FIT, $\{S_5,\Z/2\Z,e\}$. We can't have order ?? so we eliminate $e$, we can't have order 5 so we eliminate $\Z/2\Z$. Thus, the only thing is $S_5$. It's doing too many interesting things to have such a small image.
    \item We obtain an injective map from $S_5$ to $S_6$. Why do it in such a strange way? Because it also has the property that its image acts transitively on six points.
    \begin{itemize}
        \item Remark: You can restrict to $A_5\to S_6$, and we've seen this before where $A_5\cong\text{Do}$ and $S_6$ is the pairs of opposite faces.
    \end{itemize}
    \item So what we say is that we have an \textbf{exotic} subgroup $S_5$ inside $S_6$.
    \item Let's call $S_5$, $H$ now. $[S_6:H]=6$. Thus, we have $S_6\acts S_6/H$ by left multiplication. This action is transitive. $\Stab(H)=H$.
    \item $\psi:S_6\to S_{S_6/H}$.
    \item $\ker\psi=\{1,A_6,S_6\}$, $\im\psi=\{S_6,\Z/2\Z,e\}$ where we know once again that the latter two can't happen.
    \item So we get $\psi:S_6\to S_{S_6/H}\cong S_6$ is exotic??
    \begin{itemize}
        \item $H$ under this map maps to a boring $S_5$.
        \item We know that we're sending a whole bunch of shit around (see picture).
    \end{itemize}
    \item There will be a blog post on all of this nonsense.
    \item Future: Groups of order 5, groups of prime order, the Sylow theorems, and simple groups.
\end{itemize}




\end{document}