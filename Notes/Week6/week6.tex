\documentclass[../notes.tex]{subfiles}

\pagestyle{main}
\renewcommand{\chaptermark}[1]{\markboth{\chaptername\ \thechapter\ (#1)}{}}
\setcounter{chapter}{5}

\begin{document}




\chapter{???}
\section{Examples of Group Actions}
\begin{itemize}
    \item \marginnote{10/31:}Today: A number of interesting group actions.
    \item \textbf{Left action} (of $G$ on $X$): A group action of the form $g\cdot x$ (as opposed to $x\cdot g$).
    \item Let $G$ be a group, and let $X=G$. Take $g\cdot x=gx$.
    \begin{itemize}
        \item Axiom confirmation.
        \begin{enumerate}
            \item $e\cdot x=ex=x$.
            \item $g\cdot(h\cdot x)=ghx=gh\cdot x$.
        \end{enumerate}
        \item Let $e\in X$. Then $\Orb(e)=X$. In particular, this means that the action is transitive.
        \item $\Stab(x)=\{g\in G\mid gx=x\}=\{e\}$ for $x\in X$ arbitrary, in general.
        \item $\ker=\{e\}$. This also follows from the above. Thus, the action is faithful.
    \end{itemize}
    \item Corollary: Let $G$ be a finite group. Then $G$ is isomorphic to a subgroup of $S_n$ for some $n$. We may take $n=|G|$.
    \begin{itemize}
        \item Construction: We invoke the proposition from last lecture. In particular, we know that the action $G\acts G$ implies the existence of a homomorphism $\phi:G\to S_G$ defined by $g\mapsto\psi_g$.
        \item The map in the above construction has trivial kernel. By the FIT, $G/\ker\cong\im\phi$. Combining these results, we obtain $G\cong G/\ker\cong\im\phi\leq S_n$.
        \item Applying this construction to $S_3$, we deduce that $S_3\leq S_6$.
    \end{itemize}
    \item $\text{SO}(2)\cong\R/\Z\cong\Q/\Z\oplus\Q^\infty$.
    \begin{itemize}
        \item In infinite cases, you usually want to consider some other topological things that disappear in the finite case.
    \end{itemize}
    \item Let $G$ be a group and take $X=G$ again. We can also consider $g\cdot x=gxg^{-1}$.
    \begin{itemize}
        \item Axioms.
        \begin{enumerate}
            \item $e\cdot x=exe^{-1}=x$.
            \item $g\cdot(h\cdot x)=ghxh^{-1}g^{-1}=(gh)x(gh)^{-1}=gh\cdot x$.
        \end{enumerate}
        \item $\Orb(e)=\{e\}$; not transitive if $|G|>1$.
        \item Let $x\in X$. Then $\Orb(x)$ is the conjugacy class of $x$.
        \item $\Stab(x)=C_G(x)$.
        \item $\ker=Z(G)$. Thus, the group action is faithful iff the center is trivial. Abelian implies not faithful.
    \end{itemize}
    \item A nice thing about these constructions is that they cast other constructions we've encountered in the more general language of group actions.
    \item \textbf{Right actions} are even nastier than left cosets and right cosets, so Calegari will not mention them again.
    \begin{itemize}
        \item $g\cdot x=x\cdot g^{-1}$ and $g\cdot(h\cdot x)=(x\cdot h^{-1})\cdot g^{-1}$.
    \end{itemize}
    \item Let $G=G$, $X$ be the subgroups of $G$. $g\cdot H=gHg^{-1}$.
    \begin{itemize}
        \item Note that $H\leq G$ does indeed imply that $gHg^{-1}\leq G$. In particular, \dots
        \begin{itemize}
            \item $H$ is nonempty (contains at least $e$), so $gHg^{-1}\supset\{geg^{-1}\}$ is nonempty;
            \item $gh_1g^{-1},gh_2g^{-1}\in gHg^{-1}$ imply that $gh_1g^{-1}gh_2g^{-1}=g(h_1h_2)g^{-1}\in gHg^{-1}$;
            \item $ghg^{-1}\in gHg^{-1}$ has inverse $gh^{-1}g^{-1}\in gHg^{-1}$.
        \end{itemize}
        \item Axioms (entirely analogous to the last example).
        \item $\Orb(H)$ is the "conjugates" of $H$.
        \item $\Stab(H)=N_G(H)$.
        \item $\ker=?$. We know that $Z(G)\subset\ker$. The conclusion is that there is not a nice definition for the kernel other than the intersections of the stabilizers/normalizers.
        \begin{itemize}
            \item ...
            \item If any $H\triangleleft G$ is normal, and $x\in G$ had order 2, then $\gen{x}\triangleleft G$, meaning that $gxg^{-1}\in\gen{x}$, i.e., $x\in Z(G)$, so this rules out $D_8$??
        \end{itemize}
    \end{itemize}
    \item Fix $G$ and $H\leq G$. Let $X=G/H$ (not assuming $H\triangleleft G$, so we know that $G/H$ is the set of left cosets but it is not a group in general). Define $g\cdot xH=gxH$.
    \begin{itemize}
        \item We have $g\cdot xhH=gxhH$.
        \item Orbit: $\Orb(eH)=X$.
        \item Stabilizer: $\Stab(eH)=H$.
        \begin{itemize}
            \item $\Stab(gH)=gHg^{-1}$.
            \item This is because $(ghg^{-1})gH=ghH=gH$.
            \item Go to the more general case $G\acts X$, $\Stab(x)=H$. Then $gHg^{-1}\subset\Stab(g\cdot x)$??
        \end{itemize}
        \item Transitive: Yes (see orbits).
        \item Faithful: If $H$ is normal, no. If $H$ contains a normal subgroup, no. Maybe yes.
        \item Kernel: If $H$ is normal, then $\ker=H$. In general, $\ker=\bigcap_{g\in G}gHg^{-1}$ (the largest normal subgroup of $H$).
    \end{itemize}
    \item Takeaway: General constructions allow us to see things we've already done.
    \item Next time: The most useful theorem of the course, that provides lots of information on relations between objects.
\end{itemize}



\section{Orbit-Stabilizer Theorem}
\begin{itemize}
    \item \marginnote{11/2:}We will have a take-home open-book final. Should take you a couple hours or a little more to do, but we'll have more time than that. Don't Google answers or collaborate. We'll have more practice problems (and 50\% of the exam will be on that sheet); if we do every problem on the sheet, we'll certainly get an A.
    \item We will cover all theoretical material by Thanksgiving and then spend the rest of the time exploring applications.
    \item Today: The most fundamental theorem of the class.
    \item Let $G$ be a group acting on a set $X$.
    \item Theorem (Orbit-Stabilizer Theorem): Let $x\in X$ be arbitrary. Then
    \begin{equation*}
        |G| = |\Orb(x)|\cdot|\Stab(x)|
    \end{equation*}
    \begin{proof}
        We will break up $G$ and count it in two different ways.
        \begin{equation*}
            G = \bigsqcup_{y\in\Orb(x)}\{g\mid g\cdot x=y\}
        \end{equation*}
        Each of these sets is equal to $g\cdot\Stab(x)$ (the left coset of the stabilizer by $g$).\par
        Thus,
        \begin{equation*}
            |G| = \sum_{\Orb(x)}|g\cdot\Stab(x)|
            = \sum_{\Orb(x)}|\Stab(x)|
            = |\Orb(x)|\cdot|\Stab(x)|
        \end{equation*}
        as desired.
    \end{proof}
    \item Examples:
    \begin{itemize}
        \item Let $H\leq G$, $X=G/H$. Then $G$ acts on $X$ by left multiplication. Taking $x=H$ in particular, we have that
        \begin{equation*}
            |G| = |G/H|\cdot|H|
        \end{equation*}
        \item $G=S_n$, $X=[n]$.
        \begin{itemize}
            \item Then $S_n=\{\sigma(1)=1\}\cup\{\sigma(1)=2\}\cup\cdots\cup\{\sigma(1)=n\}$. This is analogous to the proof strategy decomposition.
        \end{itemize}
        \item $G$ acts on $G$ by conjugation.
        \begin{itemize}
            \item Take $g\in G$. Then $\Orb(g)=\{g\}$, i.e., the conjugacy class of $g$, and $\Stab(g)=C_G(g)$. Therefore, we have the below corollary.
        \end{itemize}
        \item $G=S_n$.
        \begin{itemize}
            \item Let $g=(1,\dots,k)$ for $2\leq k\leq n$. Recall that $|g|=n!/(n-k)!k$. Thus, $|C_{S_n}(g)|=(n-k)!\cdot k$.
            \item Alternatively, we can derive the order of this centralizer directly: $C_{S_n}(g)=\gen{g}\times S_{n-k}$, i.e., all powers of the $k$-cycle and everything that's disjoint. $\times$ denotes the direct product.
        \end{itemize}
        \item $G=S_4$, $g=(12)(34)$.
        \begin{itemize}
            \item $|\{g\}|=3$, so $|C_G(g)|=8$.
            \item Here $C_G(g)=D_8$. Visualize a square with vertices clockwise (1,4,2,3).
        \end{itemize}
        \item $G=S_6$, $g=(16)(25)(34)$.
        \begin{itemize}
            \item We have that $|\{g\}|=6!/2^3\cdot 3!=15$, so $|C_{S_6}(g)|=48$. The centralizer is the set of all elements satisfying $\sigma(i)+\sigma(7-i)=7$.
            \item Moreover, there is an injective homomorphism from $\widetilde{\text{Cu}}\hookrightarrow S_6$ whose image is exactly the centralizer of $(16)(25)(34)$. Moreover, it follows that $C_{S_6}(g)\cong S_4\times S_2$.
            \item Let $h=(16)$. Then $|\{h\}|=|\{g\}|=15$. Does there exist an automorphism of $S_6$ to $S_6$ which sends $h\to g$? No: $S_2\times S_4\cong C_{S_6}(h)$ and $C_{S_6}(g)\cong S_2\times S_4$.
        \end{itemize}
    \end{itemize}
    \item Corollary: We have that
    \begin{equation*}
        |G| = |\{g\}|\cdot|C_G(g)|
    \end{equation*}
    \item $\bm{\widetilde{\textbf{Cu}}}$: The set of all orthogonal symmetries of the cube (i.e., including reflections).
    \begin{itemize}
        \item There is an isomorphism between $\text{Cu}\times\Z/2\Z$ and $\widetilde{\text{Cu}}$ defined by $(g,1)\mapsto g$ and $(g,-1)\mapsto -g$. The reverse function is $g\mapsto(g\cdot\deg g,\deg g)$.
        \item $\widetilde{\text{Cu}}$ acts on 6 faces.
    \end{itemize}
    \item The pace will be this fast through Thanksgiving.
\end{itemize}




\end{document}